\documentclass[a4paper,USenglish]{lipics-v2021}
\usepackage{amssymb,amsthm,amsmath}
\usepackage{listings}
\usepackage{hyperref}
\usepackage{cleveref}
\usepackage{thmtools, thm-restate}

\newcommand{\moveToAppendix}{1}
\newcommand{\starBool}{0} 
\newcommand{\starCommand}{\if\starBool1{ {\normalfont($\star$)}}\fi}
\makeatletter
\def\thmhead@plain#1#2#3{%
	\thmname{#1}\thmnumber{\@ifnotempty{#1}{ }\@upn{#2\starCommand}}%
	\thmnote{ {\the\thm@notefont (#3)}}}
\let\thmhead\thmhead@plain
\makeatother
\newcommand{\starToggle}{\if\starBool0{\gdef\starBool{1}}\else{\gdef\starBool{0}}\fi}
\newenvironment{starTheorem}[1]
{\if\moveToAppendix1\starToggle\fi
	\restatable{theorem}{#1}
}
{ 
	\endrestatable
	\if\moveToAppendix1\starToggle\fi
}

\title{Automatic placing of proofs to/from the appendix}
\author{Jason {Schoeters}}{University of Cambridge \and \url{https://jschoete.github.io/}}{jason.schoeters.cs@gmail.com}{}{}
\authorrunning{J. Schoeters} %TODO mandatory. First: Use abbreviated first/middle names. Second (only in severe cases): Use first author plus 'et al.'
\Copyright{Jason Schoeters} %TODO mandatory, please use full first names. LIPIcs license is "CC-BY";  http://creativecommons.org/licenses/by/3.0/
\ccsdesc[500]{General and reference}
\keywords{Efficient formatting} %TODO mandatory; please add comma-separated list of keywords
\hideLIPIcs

\begin{document}
	
\maketitle

\begin{abstract}
	In this document, we show how to automatically place proofs to, or from, the appendix and marking corresponding theorems automatically by a $\star$ symbol, or removing it. From a user perspective, this is easily done through the switching of a 1 to a 0. This is useful when working on a document for a conference with a page limit, and afterwards easily obtain the journal version.
\end{abstract}

\section{First chapter}

In this \verb|.tex| file, scroll up until you find \verb|\newcommand{\moveToAppendix}{1}|. This command is set to 1, meaning all proofs associated with the \verb|\begin{starTheorem}| such as \Cref{theorem:two} used below (note how the proof is encased in a specific format as well), are automatically moved to the appendix. The reader is encouraged to freely change the value from 1 to 0 to observe how proofs automatically place themselves to/from the appendix, and how theorems obtain/remove the $\star$ symbol, and how the appendix is created/removed automatically as well. 

\begin{theorem}
	\label{theorem:one}
	Three colours are sufficient for an edge colouration of the complete graph of order $n$ to avoid some spanning tree of some same colour.
\end{theorem}

\begin{proof}
%		Suppose the vertices are $v_1, v_2, ..., v_n$.
		Colour edge $\{v_1, v_2\}$ with colour 1. Colour the remaining edges of $v_1$ with colour 2. Colour the rest of the edges with colour 3.
		Colour 1 does not cover vertex $v_3$, colour 2 does not cover vertex $v_2$, and colour 3 does not cover vertex $v_1$, so no spanning tree can exist of some same colour\footnote{Exactly two connected components of colour 2 and 3 exist, and $n-1$ connected components of colour 1.}.
\end{proof}

\begin{starTheorem}{starTheoremTwo}
	\label{theorem:two}
	Two colours are not sufficient for an edge colouration of the complete graph of order $n$ to avoid some spanning tree of some same colour.
\end{starTheorem}

\newcommand{\starTheoremTwoProof}{
\begin{proof}
	Suppose by contradiction that two colours are sufficient to avoid such a spanning tree.
	Thus, colour 1 does not create a spanning tree. This implies that at least two connected components exist of colour 1\footnote{This may include singleton connected components.}. All edges between such components must thus have colour 2, and these induce complete bipartite graphs. All these complete bipartite graphs are connected to each other and together span the whole vertex set, thus a spanning tree of colour 2 must exist. This is a contradiction.
\end{proof}
}\if\moveToAppendix0\starTheoremTwoProof\fi

\begin{corollary}
	Three colours are necessary and sufficient for an edge colouration of the complete graph of order $n$ to avoid some spanning tree of some same colour.
\end{corollary}

\begin{proof}
	\Cref{theorem:one} shows three colours are sufficient, and \Cref{theorem:two} implies this is necessary as well.
\end{proof}
	
\if\moveToAppendix1 
\newpage
\appendix	
\section{Auxiliary proofs}
\label{appendix:aux}

\starTheoremTwo*
\starTheoremTwoProof

\fi
		
\end{document}
